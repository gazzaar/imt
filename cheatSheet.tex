\documentclass{article}
\usepackage{amsmath}
\usepackage{amssymb}
\usepackage{array}
\usepackage{booktabs}
\usepackage[margin=1in]{geometry}

\begin{document}

\begin{table}[htbp]
\centering
\caption{Mathematical Symbols and Their Usage}
\begin{tabular}{>{\($}l<{$\)} l p{5cm} p{5cm}}
\toprule
\textbf{Symbol} & \textbf{Name} & \textbf{Description} & \textbf{Example} \\
\midrule
\{ \} & set & used to define a set & $S = \{ 1, 2, 3, 4, \ldots \}$ \\
\in & in, element of & used to denote that an element is part of a set & $1 \in \{1, 2, 3\}$ \\
\not \in & not in, not an element of & used to denote than an element is not part of a set & $4 \not \in \{1, 2, 3\}$ \\
$|S| & cardinality & used to describe the size of a set & $S = \{1, 2, 2, 2, 3, 4, 5, 5 \}$, $|S| = 5$ \\
:, \mid & such that & used to denote a condition & $\{x^2 : x + 3 \text{ is prime}\}$ \\
\subseteq & subset & set $A$ is a subset of set $B$ when each element in $A$ is also in $B$ & $A = \{ 1, 2 \}$, $B = \{ 2, 1, 4, 3, 5 \}$, $A \subseteq B$ \\
\subset & proper subset & set $A$ is a proper subset of set $B$ when each element in $A$ is in $B$ and $A \neq B$ & $A = \{ 1, 2, 3, 4, 5 \}$, $B = \{ 2, 1, 4, 3, 5 \}$ \\
\supseteq & superset & set $A$ is a superset of set $B$ when $B$ is a subset of $A$ & $A = \{ 2, 4, 6, 7, 8 \}$, $B = \{ 2, 4, 8 \}$, $A \supseteq B$ \\
\cup & union & a set with elements in set $A$ or in set $B$ & $A = \{1, 2\}$, $B = \{2, 3, 5\}$, $A \cup B = \{1, 2, 3, 5\}$ \\
\cap & intersection & a set with elements in set $A$ and in set $B$ & $A = \{1, 2\}$, $B = \{2, 3, 5\}$, $A \cap B = \{2\}$ \\
\emptyset & empty set & the set with no elements & $\{1, 2, 3\} \cap \{4, 5, 6\} = \emptyset$ \\
\setminus & set difference & elements in set $A$ that are not in $B$ & $A = \{1, 2, 3, 4\}$, $B = \{2, 3, 5, 8\}$, $A \setminus B = \{1, 4\}$ \\
\times & Cartesian product & all possible combinations of elements from $A$ and $B$ & $A = \{1, 2\}$, $B = \{3, 4\}$, $A \times B = \{(1, 3), (2, 3), (1, 4), (2, 4)\}$ \\
A^c & complement & elements of universe $U$ not in set $A$ & $U = \{1, 2, 3, 4, 5\}$, $A = \{2, 4\}$, $A^c=\{1, 3, 5\}$ \\
f : A \rightarrow B & function & maps elements of set $A$ to set $B$ & $f(x) = x^2+5$ is $f : \mathbb{R} \rightarrow \mathbb{R}$ \\
f : x \mapsto x^3 & mapping & maps any $x$ to $x^3$ & $f: x \mapsto x^2+5$ \\
\mathbb{N} & natural numbers & set of natural numbers starting at 1 & $\mathbb{N} = \{1, 2, 3, \ldots\}$ \\
\mathbb{N}_0 & whole numbers & set of whole numbers starting at 0 & $\mathbb{N}_0 = \{0, 1, 2, 3, \ldots\}$ \\
\mathbb{Z} & integers & whole numbers with their negatives & $\mathbb{Z} = \{\ldots, -3, -2, -1, 0, 1, 2, 3, \ldots\}$ \\
\mathbb{Q} & rational numbers & all $\frac{p}{q}$ where $p,q \in \mathbb{Z}, q \neq 0$ & $\{\frac{1}{2}, \frac{5}{14}, \frac{-17}{3}\} \subset \mathbb{Q}$ \\
\wedge & conjunction & $P \wedge Q$ true if both $P,Q$ true & $P = (2 \text{ is prime}), Q = (8 \text{ is cube})$ \\
\vee & disjunction & $P \vee Q$ true if either $P,Q$ true & $P = (2 \text{ is prime}), Q = (4 \text{ is square})$ \\
\neg & negation & $\neg P$ true if $P$ false & if $P = (35 \text{ is prime})$ then $\neg P$ is true \\
\implies & implication & if $P$ then $Q$ & if $P = (x \text{ div by }4)$, $Q = (x \text{ even})$ \\
\iff & if and only if & $P \implies Q$ and $Q \implies P$ & $P = (\text{new year}), Q = (\text{January }1)$ \\
\forall & for all & refers to all elements in a set & if $A = \{2, 4, 10\}$ then $x \in \mathbb{N} \text{ } \forall x \in A$ \\
\exists & there exists & at least one exists & $\exists x \in \mathbb{N}_0 : x = -x$ \\
\oplus & XOR & either $P$ or $Q$ true but not both & $P \oplus Q$ true for one Democrat \\
\bottomrule
\end{tabular}
\end{table}

\end{document}
